Preface
Introduction
    Exploratory style
    Situating the context
        The history of hypermedia
        Interactive media in general and its use for DSLs
        XIMPEL in education (benefits and limitations)

What does interactivity mean for hypermedia?
    What are the technicalities of it?
What could hypermedia contribute to online education compared to normal hypertext?
How could we improve XIMPEL?
How could we improve the quality of experience of XIMPEL and make it more engaging?
What do the current sample applications show regarding the relevant uses of hypermedia systems?

Explorations
    Command line tutorial (exploration 1)
    Parallel player (exploration 2)
    XIMPEL in React (exploration 3)
    Analytics framework (exploration 4)
    XIMPEL and time scrubbing (exploration 5)
    Example applications XIMPEL (exploration 6)

Summary of findings from explorations
Future work / Discussion
    How to expand current hypermedia systems?






http://sg.ximpelapps.nl/1610/ (arithmetic)
http://sg.ximpelapps.nl/1607/ (mud style thing)
http://sg.ximpelapps.nl/1602/ (education on physics)
http://sg.ximpelapps.nl/1601/ (minecraft)