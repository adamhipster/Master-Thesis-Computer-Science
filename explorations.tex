\chapter{Explorations}
Now that we have some idea of where the philosophical forefathers of XIMPEL came from it is time to explore! The conscious choice of having no methodology has been informed by the philosopher Feyerabend. There are two starting questions. These are: what is the relationship between XIMPEL and education? And how does XIMPEL need to be improved for it to better serve education? 

The idea of the explorations are that once an exploration is done, it may give rise to other questions that need to be explored. One exploration question came up by daydreaming about XIMPEL, this is exploration 4. I took a course in user experience design and realized how important a good user experience is. During my time as a student I noticed that a lot of students experienced frustrating moments while experiencing a XIMPEL presentation. It begged the question: is it possible to (semi)automatically detect user frustration within XIMPEL presentations? One assumption is that this type of detection may help the user experience for XIMPEL presentations. The chain of how one exploration caused another exploration happened as follows: exploration 1 (from the starting questions), exploration 2 (from exploration 1 and from the starting questions), exploration 3 (from exploration 1), exploration 5 (from exploration 2) and exploration 6 (from exploration 2). For some preservation of the creative context: what went through my mind will be a short story per exploration as the first few paragraphs.

Among other things, explorations have one key difference with experiments in that they are unfortunately not reproducible. This is a blessing and a curse. It is a blessing because I have encountered zero papers (!) that reported the reproduction of a result of another computer scientist. It is a curse, because I am admitting that a certain form of objectivity is lost. To be fair, many papers in computer science seem not to be reproducible because not enough information is presented in order to recreate a proposed system. And the nature of the field gives the impression that engineering replaces science. One could argue that working technology is a certain truth on its own, which is what science is about.

% Since I already outlined the history of XIMPEL I can answer the first question, which I will 
% The relationship between XIMPEL and education can be seen from its entire history starting from the beginning. Wanting to educate 

The non-reproducibility is not how this body of work in this thesis is relevant. Fortunately, there are other ways to contribute towards the academic discussion for any given topic. A few examples are: creating tools, new relevant examples of application regarding a particular system or simply asking questions that have not been asked but which are relevant. In the following paragraphs the specific contributions per exploration are stated.

\textbf{In exploration 1} The implementation regarding the creation of a command-line tutorial in XIMPEL is presented. The contribution of this exploration is showing the boundaries of XIMPEL when a developer discards the hypermedia ideal and solely focuses on education. The implementation of the command-line tutorial itself is a contribution to the XIMPEL framework, including its programmed server. Moreover, there is an attempt to answer the question: when is something hypermedia and when is it not?

\textbf{In exploration 2} the future work that was presented in the thesis of Stefan Bruins\cite{stefan2016} which is creating a parallel player for XIMPEL. Bruins focused on the main question whether the XIMPEL framework could be implemented with HTML5, CSS and JavaScript. In this exploration I focus on the question on how to architect and implement a parallel media player within XIMPEL. The contribution is an implementation and architecture overview of a parallel media player. Furthermore, the a parallel media player in XIMPEL creates new questions such as: what if a media item survives a subject switch (called media item subject switch survival, i.e. MISSS)? Or: how does one do timescrubbing in XIMPEL? These questions are explored in a new exploration.

\textbf{In exploration 3} I assess whether XIMPEL can be recreated with ReactJS (a front-end JavaScript library created by Facebook) in an effective manner. In this exploration I successfully and unsuccessfully recreate XIMPEL with ReactJS as opposed to plain JavaScript and jQuery. The contribution of this is most of XIMPEL recreated in ReactJS and a relative full description of the creative process of getting there. It furthermore, implements ideas from almost all explorations and has ended as a recreation of XIMPEL and also as my vision for XIMPEL, which is more web-based than the current version. For the attentive reader this is secretly both exploration 3 and the final exploration because of the messy timeline.

\textbf{In exploration 4} I explore how to (semi)automatically detect frustration of users within XIMPEL presentations. The contribution of this exploration is an implemented logging extension for XIMPEL. The logging extension captures: facial expressions, mouse clicks, mouse moves and the history of what the user already has seen. There is furthermore a conceptual description on how to detect frustration specifically for the XIMPEL framework.

\textbf{In exploration 5} I wanted to implement a time-scrubbing mechanism, but slowely and surely I realize that there are many twists and turns when it comes to time-scrubbing in hypermedia applications! It suffices to say that architecting and implementing time-scrubbing for single videos is much more straightforward than multiple videos (and other forms of media) that share a certain relationship with the other media elements. The contribution of this exploration is an analysis of which design choices one could possibly take in order to create time-scrubbing in XIMPEL. Analyzing these design choices was equally rooted in: literature, rationality and imagination.

\textbf{In exploration 6} In short story form I write about how I extend XIMPEL with one extra parameter in a method definition and show how this extension can lead to some serious media type hacking for media type developers. To showcase this, I implemented a YouTube media type that does not immediately detach from the DOM when a new subject is clicked on. The contribution of this exploration is: (1) the introduction of a new concept -- partial media type refreshes, (2) a showcase on extending media types so that this functionality becomes available and (3) an implemented YouTube media type that supports partial refreshes.

% Afsluitende alinea? Structuur per exploration
In closing, the structure of any exploration is as follows:
\begin{itemize}
    \item short story what went through my mind
    \item introduction to exploration
    \item the exploration itself
    \item conclusion and future work
\end{itemize}