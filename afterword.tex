Overigens, voor de LaTeX liefhebbers.

Ik heb een popular question op TeX Stack Exchange! De vraag was gesteld in het kader van mijn thesis.

In mijn thesis zal ik in de reference list de pagina's uitprinten waar ik het als short citation gebruik.

https://tex.stackexchange.com/questions/424779/can-latex-remember-from-which-page-the-user-jumped-when-clicking-on-a-reference

Met hartelijke groet,
Melvin Roest

=====
Hallo!

Korte update: met XIMPEL kun je resources razendsnel annoteren. Dit kan handig zijn vanwege meerdere redenen. In het volgende YouTube filmpje geef ik een voorbeeld m.b.t. een educationele website www.sqlteaching.com

De demo loopt van ongeveer minuut 2 tot en met minuut 5. Ik heb geen server online staan, anders had ik de XIMPEL presentatie daarop gezet en gelinkt. De annotaties die je ziet zijn daadwerkelijke tips waarvan ik had gewild dat ze in de website zelf zaten.

https://www.youtube.com/watch?v=nwdd6zL4JGQ -- de gehele video (9 min.) is een update hierover.

Met hartelijke groet,
Melvin Roest

P.S. near-immediate future work hiervoor (ga ik waarschijnlijk niet doen maar who knows)

Dit zou nog uitgebouwd kunnen worden (future work). Ik weet niks over eerder werk gedaan in dit veld, maar een hypermedia framework + iframe functionaliteit komt erg ver. Ik zie voornamelijk nog het idee van pijlen tekenen. Het probleem met pijlen op dit moment is dat XIMPEL een x, y coordinaten systeem heeft en je in een text-editor niet precies weet waar zo'n pijl dan precies komt. Mensen moeten dingen visueel zien als positionering een beetje erg involved wordt.

Een mogelijke oplossing: wat je hiervoor zou kunnen doen (om maar van Anton zijn ideeën stevig te lenen, aangezien ik dit moest doen tijdens Multimedia Authoring) is een simpele tekentool maken die geïntegreerd kan worden met XIMPEL for development reasons. Hierop kun je een pijl tekenen (or anything else) en dan kun je de coordinaten van de pijl / tekening in SVG exporteren en importeren als een image in XIMPEL.
=====