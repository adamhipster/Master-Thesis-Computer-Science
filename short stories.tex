% MOOCs / Terminal
I envisioned a great Massive Online Open Course (MOOC) before me. I marveled looking at the Machine Learning MOOC on Coursera or the Introduction to Computer Science by Harvard at edX. I wondered: can XIMPEL do this? The test forms are: quizzes and programming assignments. "XIMPEL can already do quizzes," I thought. But that is the only assessment form it has. A lot of computer science MOOCs have programming assignments as well. ``How would that work in XIMPEL?`` I thought. This thought coincided with a frustration: I never had a command-line tutorial when I studied at the Vrije Universiteit Amsterdam. Furthermore, with a command-line comes access to programming in a lot of languages.

I craved creating a command-line tutorial so much that I immediately skipped every logical step and began straight away with the implementation. The most reasonable step would to start with design and asking myself whether creating a command-line tutorial is a good idea. But I chose the route of a passionate lover.  

In its implementation, I stumbled and almost fell. This command-line, is a media type from hell. Was it a media type or is it an application? It did not matter, its completion heightened my adoration. For XIMPEL can now compete with Coursera: not in design but in pragmaticism, not in flexibility but in programming simplicity, not in comprehensiveness but in development speed. With this extension, XIMPEL occupies a unique niche of hypermedia and education.

% Parallel player
I read Stefan his thesis over and over. While a thesis is academic, I detected a form of sadness in it, a regret perhaps. There was a longing to play media items in parallel. Everything was set for it: the architecture expected it, in Stefan's thesis it was written about and because of the command-line tutorial, I needed it. However, it was not there.

At the time I did not think much of it, but at the end of my thesis I realized that parallel media playback and parallel media items interfacing with applications is a topic of research for computer scientists that could provide new insights into the nature of computation and human-computer interaction. Moreover, parallel media playback from a hypermedia standpoint is inevitably linked with multiple time lines. Because of this it has some resemblance to time travel or the simulation of time travel. The cool thing is that a form of time travel can be simulated within XIMPEL! But I am getting ahead of myself.

% XIMPEL in ReactJS
ReactJS is hip, new and shiny. It introduces a relatively unknown paradigm to web developers called reactive programming. I wanted to work in this language because the web community seems to settle on this as a best practice method of creating web applications. Moreover, if my thesis supervisor can demand that I work on his framework pure for promotional reasons, then I can decide I learn a new framework for self promotional reasons. Later on I also justified the use of React academically but I do not want to shy away from the inherent selfishness that academia has. If science is a form of truth discovery, then the truth of its process should not be hidden.

Related to this is that I wanted to create the same type of exploration that Stefan Bruins did. If Stefan Bruins can port XIMPEL to JavaScript and call it academic, then I can do the same for porting the JavaScript version of XIMPEL to React. I believe in both cases the research question of ``is it possible?'' could be answered a priori with a ``yes.'' However, like Stefan I am an empiricist and the strongest form of evidence is through physical demonstration. Moreover, I altered the research focus to whether XIMPEL written in ReactJS has more advantages than XIMPEL written in JavaScript + jQuery. That answer is a lot tougher to answer a priori and hence an actual implementation, including a write up of the whole implementation experience is needed.